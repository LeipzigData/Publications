\documentclass{beamer}
\usepackage{ngerman}
\usepackage[utf8]{inputenc}
% Designelemente
\usetheme{Boadilla}
\beamertemplatenavigationsymbolsempty

\newenvironment{code}{\footnotesize\tt \begin{tabbing}
\hskip12pt\=\hskip12pt\=\hskip12pt\=\hskip12pt\=\hskip5cm\=\hskip5cm\=\kill}
{\end{tabbing}}

\setbeamertemplate{frametitle}{%
\vspace{-0.165ex}

\begin{beamercolorbox}[wd=\paperwidth,dp=1ex, ht=4.5ex, sep=0.5ex,
    colsep*=0pt]{frametitle}
  \usebeamerfont{frametitle}   \strut \insertframetitle  \hfill 
  \raisebox{0ex}[0pt][-\ht\strutbox ]{
    \begin{minipage}[b]{.4\textwidth}\raggedleft \tiny Leipzig Open Data
      Initiative\\ Gefördert von der Stadt Leipzig
  \end{minipage}}
\end{beamercolorbox}%
}%

\title[Jugendstadtplan]{Teilprojekt Jugendstadtplan}

\subtitle{Bericht zum Abschlusstreffen des Projekts\\ „Leipzig Open Data
  Initiative“}

\author[Gräbe]{Prof. Dr. Hans-Gert Gräbe}

\institute[Uni Leipzig]{Leipzig Open Data Team\\
  \texttt{http://leipzig-data.de}} 

\date{Leipzig, 26. April 2013}
\begin{document}
\begin{frame}
\maketitle
\end{frame}
\begin{frame}{Hintergrund}{}
Teilprojekt, das sich im Sinne von Open Innovation in der Arbeit nach der
Ideenbörse herauskristallisiert hat.\bigskip
\begin{itemize} 
\item \textbf{Zielstellung:} Präsentation eines Ausschnitts der verfügbaren
  Informationen im Kontext der WorldSkills 2013 Anfang Juli 2013 in Leipzig. 
\item \textbf{Partner:} MINT-Netzwerk Leipzig, Arbeitsgruppe, die im Auftrag
  der Stadt Leipzig ebenfalls an einem Jugendstadtplan arbeitet. 
\end{itemize}
\end{frame}

\begin{frame}{Erreichte Ergebnisse}{}
\begin{itemize} 
\item Prototypische Lösung auf der Basis einer OpenStreetMap-Karte\\ $\to$
  \url{http://leipzig-data.de/Jugendstadtplan}.
\item Extraktion und Aufbereitung der zu MINT-Orten verfügbaren Informationen.
\item Erste Auswertung weiterer Datenquellen, insbesondere OpenStreetmap
  Leipzig. 
\item Fortführung als Projektpraktikum mit einer Gruppe von Studierenden im
  Rahmen der Lehrveranstaltung „Kreativität und Technik“ im Sommersemester
  2013.
\item Die Stadt Leipzig hat das Interesse bekräftigt, Informationen zu
  MINT-Orten auch in deren Jugendstadtplan aufzunehmen.
\end{itemize}
\end{frame}
\end{document}

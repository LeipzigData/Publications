\documentclass{beamer}
\usepackage{ngerman}
\usepackage[utf8]{inputenc}
% Designelemente
\usetheme{Boadilla}
\beamertemplatenavigationsymbolsempty

\newenvironment{code}{\footnotesize\tt \begin{tabbing}
\hskip12pt\=\hskip12pt\=\hskip12pt\=\hskip12pt\=\hskip5cm\=\hskip5cm\=\kill}
{\end{tabbing}}

\setbeamertemplate{frametitle}{%
\vspace{-0.165ex}

\begin{beamercolorbox}[wd=\paperwidth,dp=1ex, ht=4.5ex, sep=0.5ex,
    colsep*=0pt]{frametitle}
  \usebeamerfont{frametitle}   \strut \insertframetitle  \hfill 
  \raisebox{0ex}[0pt][-\ht\strutbox ]{
    \begin{minipage}[b]{.4\textwidth}\raggedleft \tiny Leipzig Open Data
      Initiative\\ Gefördert von der Stadt Leipzig
  \end{minipage}}
\end{beamercolorbox}%
}%

\title[Gelbe Seiten]{Teilprojekt Gelbe Seiten}

\subtitle{Bericht zum Abschlusstreffen des Projekts\\ „Leipzig Open Data
  Initiative“}

\author[Gräbe]{Prof. Dr. Hans-Gert Gräbe}

\institute[Uni Leipzig]{Leipzig Open Data Team\\
  \texttt{http://leipzig-data.de}} 

\date{Leipzig, 26. April 2013}
\begin{document}
\begin{frame}
\maketitle
\end{frame}
\begin{frame}{Vorhaben (Ergebnis Ideenbörse)}{}
\begin{itemize} 
\item \textbf{Zielstellung:} Aufbau eines Systems \emph{gelber Seiten} für die
  Stadt Leipzig, um Akteure überhaupt namentlich benennen zu können.
\item \textbf{öffentlich:} Aufbau und Fortentwicklung eines Datenbestands
  dieser Akteure sowie allgemein anerkannter Informationen zu diesen in
  Verantwortung der Leipziger Initiative für Offene Daten. 
\item \textbf{privat:} Anreicherung dieser Informationen um private Ansichten
  und Wertungen zu regionalen Prozessen und Entwicklungen in privater
  Verantwortung.
\item \textbf{Bezugspunkte} in der bisherigen Diskussion: API Leipzig
  (Medienhandbuch), Zukunftsakademie (Werner Stickler), MINT-Netzwerk Leipzig
  (Prof. H.-G. Gräbe), Branchenbuch Energie und Umwelttechnik des Netzwerks
  Energie und Umwelt, Vereinsdatenbank der Stadt Leipzig, sportinleipzig.de 
\end{itemize}
\end{frame}

\begin{frame}{Ergebnisse}{}

Beispielanfrage nach Beschreibungen von MINT-Orten\\ (Quelle: Projekt
Jugendstadtplan) 

\begin{code}
  PREFIX ldtag: <http://leipzig-data.de/Data/Tag/>\\
  PREFIX ld: <http://leipzig-data.de/Data/Model/>\\
  PREFIX rdfs: <http://www.w3.org/2000/01/rdf-schema\#>\\
  SELECT ?l ?k WHERE \{\+\\
    ?p ld:hasTag ldtag:MINT .\\
    ?p a ld:Ort .\\
    ?p rdfs:label ?l .\\
    ?p ld:Kurzinformation ?k .\-\\
  \}
\end{code}
\end{frame}

\begin{frame}{Ergebnisse}{}
\begin{itemize} 
\item Konzeptionelle Konsolidierung der Grundschemata
\begin{itemize} 
\item Aktivitätsebene: Orte, Träger, Personen
\item Angebotsebene: Orte, Projekte, Angebote, Events
\end{itemize}
\item Aufsetzen eines Ergänzungs- und Qualifizierungsprozesses des
  Datenbestands im Kontext des Teilprojekts \emph{LD.Events}.
\item Übernahme und Einbindung ausgewählter Datenbestände von Partnern (Hosts
  und Venues von API Leipzig, Daten aus dem Kontext des MINT-Netzwerks). 
\end{itemize}
\end{frame}
\end{document}

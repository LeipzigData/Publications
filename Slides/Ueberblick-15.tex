\documentclass{beamer}
\usepackage{ngerman}
\usepackage[utf8]{inputenc}

% Designelemente
%\usetheme{Boadilla}
\usetheme{Hannover}
\beamertemplatenavigationsymbolsempty

\title{Auf dem Weg zu Leipzig Data} 

\subtitle{Das Leipzig Data Projekt}

\author[Leipzig Data Team]{Prof. Dr. Hans-Gert Gräbe\\ Leipzig Data Team}

\institute[]{Universität Leipzig\\ \url{http://leipzig-data.de}}

\date{Version vom 14. Februar 2015}
\begin{document}
\begin{frame}
\titlepage
\end{frame}

\section{Der Weg zu Leipzig Data}
\begin{frame}{Die Herausforderung}{}\small
  \begin{itemize}
  \item Web 2.0 als tief greifender technologischer Wandel digitaler
    Repräsentationsformen, der seit etwa 2005 deutlich Fahrt aufgenommen hat
    und die Kommunikationsstrukturen der Gesellschaft grundlegend umkrempeln
    wird.
  \item Basistechnologie eines neuen Technologie-Zyklus, dessen
    Durchdringungsphase etwa 2008 begonnen hat und als typische pervasive
    Technologie kaum einen gesellschaftlichen Bereich auslassen wird.
  \item Dieser Herausforderung muss sich Regionalentwicklung stellen.
  \end{itemize}
\end{frame}

\begin{frame}{Das Internet}{}
  \begin{itemize}
  \item Web der Daten? Daten werden von {Akteuren} zu von ihnen verfolgten
    {Zielen} zusammengetragen. 

    99\,\% der Webseiten im Internet enthalten {Texte} bzw.\ {textuelle
      Repräsentationen}.
  \item Web der Worte! Web der Sätze! Web der Stories! 

    Mit den neuen Technologien rückt {dieser} Aspekt gegenüber einem engen
    Begriff {Daten} in den Vordergrund.
  \item Aber (Faust): {\ldots} kann ich das \emph{Wort} so hoch unmöglich
    schätzen {\ldots}
  \item Zusammen mit den {Worten} müssen die {Akteure} und ihre
    {kooperativen unternehmerischen Taten} im Blick bleiben.
  \end{itemize}
\end{frame}

\section{Die Vision}
\begin{frame}{Leipzig Data – Die Vision}{}
  \begin{itemize}
  \item Mit dem Auf- und Ausbau breitbandiger leistungsfähiger digitaler
    Kommunikationsnetze sind die Grundlagen gegeben, unser gesamtes soziales
    Leben auf dieser erweiterten Basis neu zu strukturieren.
  \item {Leipzig Data} ist mehr als {Leipzig Open Data}, letzteres
    aber dessen Kern.
  \item {Leipzig Open Data} ist viel mehr als {Offene Daten der
    Stadt}, wir müssen (inhaltlich) nicht auf die Stadtverwaltung warten, um
    loszulegen.
  \item Im Mittelpunkt von Leipzig Data stehen nicht die Daten (bzw.\ Worte)
    selbst, sondern die Menschen und deren Motive, die auf dieser Grundlage
    {planen} und {handeln}.
  \end{itemize}
\end{frame}

\section{Wo stehen wir?}
\begin{frame}{Wo stehen wir?}{}\small
  \begin{itemize}
  \item Vorarbeiten von API Leipzig (Orte und Adressen in Leipzig, Übersicht
    über Medienunternehmen, Event-Daten) sowie aus dem MINT-Netzwerk
    (Angebote, MINT-Orte, Träger, einige Bereiche der Jugendhilfe) sind
    weitgehend aufgearbeitet, in die Strukturen von Leipzig Data integriert,
    weiter angereichert und fortgeschrieben.
  \item Kurzzeitprojekt „Leipzig Open Data Initiative“ im Rahmen der
    städtischen Ausschreibung Open Innovation Nov. 2012 bis April 2013.
  \item Einbindung in die studentische Ausbildung durch Seminararbeiten und 
    Projektpraktika. Damit Erschließung weiterer Datenbestände sowie
    Konsolidierung von Technologien.
  \item Zusammenarbeit mit dem OK Lab Leipzig --
    \url{http://codefor.de/leipzig/}.
  \end{itemize}
\end{frame}

\section{Längerfristige Teilprojekte}
\begin{frame}{Projekt „Gelbe Seiten“}{}\small
  \begin{itemize}
  \item Aus APILeipzig wurden über 65\,000 Leipziger Adressen übernommen, nach
    einem Standard in URIs verwandelt und mit Geodaten angereichert. Damit
    lassen sich Orte in Leipzig referenzieren.
  \item Dazu wurden (aktuell 428) Orte identifiziert, mit URI versehen und
    einer Adresse zugeordnet, zu denen in Leipzig Data „Geschichten“ erzählt
    wurden.
  \item Siehe \url{http://www.leipzig-data.de/gelbe-seiten/}.
  \item Zu diesen Orten wurden unsystematisch weitere standardisierte
    Informationen zusammengetragen.
  \end{itemize}
In verschiedenen zeitlich beschränkten Projekten werden auf der Basis
\emph{Ortsinformationen} zusammengetragen, die Informationen zum jeweiligen Ort
unter einem bestimmten Gesichtspunkt zu einer bestimmten Zeit darstellen, etwa
das Projekt „Jugendstadtplan 2013“ im Vorfeld der Worldskills. 
\end{frame}

\begin{frame}{Projekt „Event Framework“}{}\small
  \begin{itemize}
  \item \textbf{Ziel} dieses Teilprojekts ist es, eine Infrastruktur
    aufzubauen, in die Event-Daten in einheitlichem RDF-Format aus
    verschiedenen Quellen und von verschiedenen Akteuren eingespeist werden und
    der Allgemeinheit zum Gebrauch zur Verfügung stehen.
  \item Nach diesem Schema werden Event-Informationen aus verschiedenen Quellen
    (APILeipzig, Netzwerk Energie \& Umwelt, MINT-Netzwerk) zusammengetragen
    und wöchentlich aktualisiert. Events werden 90 Tage nach Ablauf aus der
    Infrastruktur gelöscht.
  \item Mit diesen Event-Informationen werden unsere Daten über Orte und
    Akteure in Leipzig weiter konsolidiert.
  \item Die prinzipiellen Möglichkeiten dieses Event Frameworks demonstriert
    unser \emph{Leipzig Data Event Widget}.
  \item Siehe \url{http://www.leipzig-data.de/events/}.
  \end{itemize}
\end{frame}

\begin{frame}{Projekt „Geolokale Informationen“}{}\small
\begin{itemize}
  \item Eine der großen Killerapplikationen Offener Daten ist die Zuordnung von
    Geokordinaten zu konkreten Aktionen und Beschreibungen und die kartenmäßige
    Darstellung dieser geolokalen Informationen.
  \item Mit dem Teilprojekt \emph{Gelbe Seiten} sind hierfür die erforderlichen
    Informationen für Anwendungsprojekte verfügbar.
  \item Es wurden verschiedene kürzere Projekte (Jugendstadtplan, Karte der
    Standorte von EEG-Anlagen, Gentrifizierungsprojekt) umgesetzt, um die
    technische Basis für derartige Kartendarstellungen zu konsolidieren --
    siehe \url{https://github.com/LeipzigData/Karten}.
\end{itemize}
\end{frame}

\section{Weitere studentische Projekte}
\begin{frame}{Weitere studentische Projekte}{}\small
  \begin{itemize}
  \item Gentri-14 -- Infomationen zu Gentrifizierungsprozessen in Leipzig.
    Siehe \url{http://www.leipzig-data.de/Gentri-14/}
  \item Energie-13 -- Dezentralen Energieerzeugungsanlagen in Leipzig.  Siehe
    \url{http://www.leipzig-data.de/Energie-13/}
  \item JSP-13 -- Jugendstadtplan von Leipzig.  Siehe
    \url{http://www.leipzig-data.de/Jugendstadtplan/}
  \end{itemize}
\end{frame}

\section{Leipzig Data und RDF}
\begin{frame}{RDF -- Grundlegende Konzepte}{}\small
\begin{itemize}
\item RDF = Resource Description Framework
\begin{itemize}\small
\item Zentrale Idee: Portioniere Informationen in elementaren Drei-Wort-Sätzen,
  speichere diese als Tripel und verwende standardisierte Werkzeuge zu deren
  Management.
\end{itemize}
\item \emph{Resources:} URI (Unique Resource Identifier), HTTP-Zugriff
\begin{itemize}\scriptsize
\item Zugriff auf weltweit verteilte Daten nach einheitlichem Prinzip.
\end{itemize}
\item \emph{Resource Descriptions:} Liefere auf eine HTTP-Anfrage eine
  angemessene Portion Information in strukturiertem RDF-Format aus, die mit
  Informationen aus anderen Quellen zu neuen RDF-Sätzen kombiniert werden kann
  (Linked Data Prinzip).
\item (Federated) Query Language SPARQL als RDF-Anfragesprache.
\item \emph{RDF Triple Stores} und \emph{SPARQL Endpunkte} als Teil einer
  weltweiten verteilten Infrastruktur von Datenspeichern.
\end{itemize}
\end{frame}

\begin{frame}{RDF und die Leipzig Data Infrastruktur}{}\small
\begin{itemize}
\item Tools und Daten sind in mehreren Repos über den github-Acoount
  \url{http://github.com/LeipzigData} öffentlich zugänglich. 
\item Grundlegende Konzepte und Ansätze werden im Wordpress-basierten Portal
  \url{http://leipzig-data.de} genauer dargestellt.
\item Ein zentraler Virtuoso basierter RDF Triple Store
  \url{http://leipzig-data.de/Data} mit eigenem SPARQL Endpunkt
  \url{http://leipzig-data.de:8890/sparql} arbeitet nach Linked Data
  Prinzipien.
\item Regelmäßige Dumps der RDF-Daten im Turtle-Format
\item Anleitung zur Installation der Tools und Daten in einem Virtuoso RDF
  Store innerhalb eines lokalen Apache Webservers.
\end{itemize}
\end{frame}

\begin{frame}{Leipzig Data Datentrukturen}\small
\begin{itemize}
\item AdministrativeGliederung.ttl -- Ortsteile und Stadtbezirke in Leipzig 
\item Adressen.ttl -- Leipziger Adress-Daten
\item API-Referenzen.ttl -- Zuordnung von URIs zwischen Leipzig Data und
  APILeipzig  
\item GeoDaten.ttl -- GeoDaten zu den Adress-Daten
\item Events.ttl -- Dump der aktuell verfügbaren Event-Informationen
\item Orte.ttl -- Orte in der Stadt Leipzig, die in Leipzig Data Geschichten
  schon einmal eine Rolle gespielt haben
\item Personen.ttl -- Personen, die in Leipzig Data Geschichten schon einmal
  eine Rolle gespielt haben
\item Strassenverzeichnis.ttl -- Leipziger Straßenverzeichnis einschließlich
  Straßennummern der Stadtverwaltung. 
\end{itemize}
\end{frame}

\section{Verfügbare Infrastruktur}
\begin{frame}{Verfügbare Infrastruktur (Zusammenfassung)}{}\small
\begin{itemize}\itemsep0pt
\item \url{http://www.leipzig-data.de} – Das Leipzig Data Portal.
\item \url{http://www.leipzig-data.de/Data} – RDF Data Store 
\item \url{http://leipzig-data.de:8890/sparql} – SPARQL Endpunkt für Anfragen
  auf den Daten.
\item \url{https://github.com/LeipzigData} – github Organisationsaccount von
  Leipzig Data mit den Projekten
  \begin{itemize}
  \item RDFData – RDF-Datenbasis
  \item Tools – eine Reihe von RDF-Werkzeugen (für Entwickler)
  \item Karten – Beispiele für geolokale Darstellungen von Daten
  \item Publications -- Veröffentlichungen und Präsentationen zum Projekt
  \end{itemize}
\item \url{http://www.leipzig-data.de/Backups} – wöchentliche Dumps der
  Daten.
\end{itemize}
\end{frame}
\end{document}

\begin{frame}\frametitle{An RDF based Road Map to a CASN}
How to reach such a goal with RDF based semantic technologies?
\begin{itemize}
\item Main idea: Turn passive users into active ones.
\item Identify and shape appropriate ontologies. 
\item Collect RDF data of such types, link to other sources along the Linked
  Data Principles.

  A very first prototype is used to collect such information and to display
  it within the Wordpress based CAFG site.
\item The stakeholders understand, that this is a techno-social, and even more
  a social than a technical process that is best discussed on the Symbolicdata
  Mailing list.
\item The CASN germ at \url{http://symbolicdata.org/casn} matures thanks to
  common efforts.
\end{itemize}
\end{frame}

\begin{frame}\frametitle{What is already done? }\small

  \begin{center}
    \url{http://symbolicdata.org/casn/FOAF-Profiles/}
  \end{center}

Basic information about People -- more than 700 instances \texttt{foaf:Person}
instances (i.e., passive users) from different sources.  Partly extended to
FOAF profiles and used to display people from the CAFG Board within the
Wordpress based CAFG site.

\begin{center}
  \includegraphics[width=.8\textwidth]{cicm-14/People.png}
\end{center}
\end{frame}

\begin{frame}\frametitle{What is already done?}\small

  \begin{center}
    \url{http://symbolicdata.org/casn/WorkingGroups/}
  \end{center}

Standard information about CA Working Groups -- 17 Instances of RDF type
\texttt{foaf:Group} and \texttt{sd:WorkingGroup} from the old CAFG site.  Used
to display that within the Wordpress based CAFG site.

\begin{center}
  \includegraphics[width=.8\textwidth]{cicm-14/WorkingGroups.png}
\end{center}
\end{frame}

\begin{frame}\frametitle{What is already done?}\small

  \begin{center}
    \url{http://symbolicdata.org/casn/SPP-Projekte/}
  \end{center}

Standard information about CA Projects -- 60 instances of RDF type
\texttt{sd:Project}, compiled from the list of projects within the SPP 1489
priority program.

\begin{center}
  \includegraphics[width=.8\textwidth]{cicm-14/Projekte.png}
\end{center}
\end{frame}

\begin{frame}\frametitle{What is already done?}\small

  \begin{center}
    \url{http://symbolicdata.org/casn/UpcomingConferences/}
  \end{center}

Information about CA conferences -- 12 instances of
\texttt{sd:UpcomingConference} and 58 instances of
\texttt{sd:PastConference}, compiled from different sources.  Used as input
for the printed version of the CA Rundbrief.

\begin{center}
  \includegraphics[width=.8\textwidth]{cicm-14/ConferenceAnnouncements.png}
\end{center}
\end{frame}

\begin{frame}\frametitle{What is already done?}\small

  \begin{center}
    \url{http://symbolicdata.org/casn/Dissertationen/}
  \end{center}

Information about dissertations in CA -- 28 instances of RDF type
\texttt{bibo:Thesis}, compiled from the CA Rundbrief.

\begin{center}
  \includegraphics[width=.8\textwidth]{cicm-14/Dissertationen.png}
\end{center}
\end{frame}

\begin{frame}\frametitle{What is already done?}\small

  \begin{center}
    \url{http://symbolicdata.org/casn/CAR-Beitraege/}
  \end{center}

Information about articles in the CA Rundbrief -- 75 instances of RDF type
\texttt{sd:Reference} to be displayed at the website of the German Fachgruppe. 

\begin{center}
  \includegraphics[width=.8\textwidth]{cicm-14/Rundbrief.png}
\end{center}
\end{frame}

\begin{frame}\frametitle{What is already done?}\small

  \begin{center}
    \url{http://symbolicdata.org/casn/News/}
  \end{center}

A first approach to Annotated News -- 2 instances of RDF types
\texttt{sioc:BlogPost} and \texttt{bibo:Document} related to blog posts on the
website of the German Fachgruppe.
\vskip6em
\begin{center}
  No picture -- pure harvesting functionality\\ to be used with SPARQL querying. 
\end{center}
\end{frame}

\section{Links}
\begin{frame}\frametitle{Links}\small
\begin{itemize}
\item \texttt{http://wiki.symbolicdata.org} -- the SD Wiki
\item \texttt{http://symbolicdata.org/XMLResources} -- the SD XML Resources
\item \texttt{http://symbolicdata.org/RDFData} -- the SD RDF Data Turtle Files
\item \texttt{http://symbolicdata.org/Data} -- the SD OntoWiki view on the
  basic RDF data
\item \texttt{http://symbolicdata.org/casn} -- the SD OntoWiki view on the
  CASN RDF data
\item \texttt{https://github.com/symbolicdata} -- the SD Repository at github
\end{itemize}
\end{frame}

\end{document}


Im Web verfügbar
● http://www.leipzig-data.de – Leipzig Data Portal
● http://leipzig-data.de:8890/sparql – Sparql Endpunkt für Anfragen
auf den Daten
● https://github.com/LeipzigData – github Organisationsaccount von
Leipzig Data mit den Projekten
● RDFData – RDF-Datenbasis
● Tools – eine Reihe von Werkzeugen (für Entwickler)
● http://www.leipzig-data.de/Data – Ontowiki zur Inspektion der Daten
● http://www.leipzig-data.de/Backups – wöchentliche Dumps der
Daten
● http://www.leipzig-data.de/ld-seminar – Seminar Semantische
Technologien als Arbeitsseminar
Hans-Gert Gräbe
Der Weg zu Leipzig Data
15

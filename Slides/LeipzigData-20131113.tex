\documentclass{beamer}
\usepackage{ngerman}
\usepackage[utf8]{inputenc}

% Designelemente
\usetheme{Boadilla}
%\usetheme{Hannover}
\beamertemplatenavigationsymbolsempty

\title{Auf dem Weg zu Leipzig Data} 

\subtitle{Präsentation für die IT- und E-Government-\\ Projektkonferenz der
  Stadt Leipzig}

\author[Das Leipzig Data Team]{Prof. Dr. Hans-Gert Gräbe\\ Leipzig Data Team}

\institute[]{Universität Leipzig\\ \url{http://leipzig-data.de}}

\titlegraphic{\scriptsize Die „Leipzig Open Data Initiative“ wurde von
  Nov. 2012 bis April 2013\\ im Rahmen des EU-Projekts „Creative Cities“ von
  der Stadt Leipzig gefördert.}

\date{Version vom 13.11.2013}
\begin{document}
\begin{frame}
\titlepage
\end{frame}

\section{Der Weg zu Leipzig Data}
\begin{frame}{Die Herausforderung}{}
  \begin{itemize}
  \item Web 2.0 als tief greifender technologischer Wandel digitaler
    Repräsentationsformen, der seit etwa 2005 deutlich Fahrt aufgenommen hat
    und die Kommunikationsstrukturen der Gesellschaft grundlegend umkrempeln
    wird.
  \item Basistechnologie des neuen Kondratjew-Zyklus, dessen technologische
    Durchdringungsphase etwa 2008 begonnen hat und als typische pervasive
    Technologie kaum einen gesellschaftlichen Bereich auslassen wird.
  \item Dieser Herausforderung muss sich Regionalentwicklung stellen.
  \item Dazu \url{http://hg-graebe.de/EigeneTexte/Roth-12.pdf}
  \end{itemize}
\end{frame}

\begin{frame}{Das Internet}{}
  \begin{itemize}
  \item Web der Daten? Daten werden von {Akteuren} zu von ihnen verfolgten
    {Zielen} zusammengetragen. 99\,\% der Webseiten im Internet enthalten
    {Texte} bzw.\ {textuelle Repräsentationen}.
  \item Web der Worte! Web der Sätze! Web der Stories! Mit den neuen
    Technologien rückt {dieser} Aspekt gegenüber einem engen Begriff
    {Daten} in den Vordergrund.
  \item Aber (Faust): {\ldots} kann ich das {Wort} so hoch unmöglich
    schätzen {\ldots} Am Anfang war die {Tat}.
  \item Zusammen mit den {Worten} müssen die {Akteure} und ihre
    {kooperativen unternehmerischen Taten} im Blick bleiben.
  \end{itemize}
\end{frame}

\begin{frame}{Das strategische Dreieck technologischen Wandels}{}
  \begin{itemize}
  \item Technologischer Wandel ist auch eine Herausforderung an die einzelnen
    Unternehmen, aber vor allem eine Frage der Entwicklung regionaler
    Standorte.
  \item „Wer zu spät kommt, den bestraft das Leben“. Was bedeutet das für den
    regionalen Wirtschaftsstandort Leipzig in dieser Frage?
  \item Technologischer Wandel auf Standort-Ebene vollzieht sich als Wandel im
    strategischen Dreieck von
    \begin{itemize}
    \item Unternehmen der Wirtschaft,
    \item Strukturen der Regionalentwicklung und Wirtschaftsförderung und
    \item regionalen akademischen Strukturen, forschungsstarke Unternehmen
      eingeschlossen.
    \end{itemize}
  \end{itemize}
\end{frame}

\begin{frame}{Das strategische Dreieck technologischen Wandels}{}
  \begin{itemize}
  \item In den Unternehmen der Wirtschaft müssen die neuen Technologien
    praktisch Fuß fassen (taktische Dimension).
  \item Strukturen der Regionalentwicklung und Wirtschaftsförderung begleiten
    und stimulieren diese Prozesse (strategische Dimension).
  \item Akademische Strukturen und forschungsstarke Unternehmen garantieren und
    koordinieren den wissensmäßigen Input für diese Veränderungsprozesse
    (inhaltliche Dimension).
  \end{itemize}
Detaillierter hierzu im Erfahrungsbericht des Projekts „Leipzig Open Data“
unter \url{http://leipzig-data.de/Upload/Erfahrungsbericht.pdf}
\end{frame}

\begin{frame}{Technologie und Wertschöpfung}{}\small
\textbf{Primäre Wertschöpfungsebene:} Unternehmen und Einrichtungen, die
materielle Produkte und Dienstleistungen unmittelbar erzeugen und dafür eine
angemessene Infrastruktur vorhalten und reproduzieren, insbesondere eine
IT-Infrastruktur.
\begin{quote}
  Bereich, in dem sich der aktuelle technologische Wandel auswirkt.
\end{quote}
\textbf{Sekundäre Wertschöpfungsebene:} Unternehmen und andere Dienstleister,
die derartige Infrastrukturen entwickeln, warten und betreiben, insbesondere
Unternehmen der IT-Branche.
\begin{quote}
  Bereich, in dem sich der aktuelle technologische Wandel vollzieht.
\end{quote}
\textbf{Tertiäre Wertschöpfungsebene:} Unternehmen und andere Dienstleister,
die Methoden, Konzepte und Modelle für diese infrastrukturellen Prozesse
entwickeln und damit die Change-Prozesse auf der zweiten Wertschöpfungsebene
triggern.
\begin{quote}
  Kern der Hi-Tech-Fähigkeit einer Region, aus dem heraus dieser technologische
  Wandel stimuliert werden kann.
\end{quote}
\end{frame}

\section{Die Vision}
\begin{frame}{Leipzig Data – Die Vision}{}
  \begin{itemize}
  \item Mit dem Auf- und Ausbau breitbandiger leistungsfähiger digitaler
    Kommunikationsnetze sind die Grundlagen gegeben, unser gesamtes soziales
    Leben auf dieser erweiterten Basis neu zu strukturieren.
    \begin{itemize}
    \item[]\footnotesize\it „Unsere Zeit bietet wie keine andere eine gewaltige
      Sammlung von Wissen in Textform dar. Die gesamte Geistesgeschichte der
      Menschheit wird auf CD-Roms, auf Internet-Seiten, in Antiquariaten und im
      Buchhandel dargeboten, alles ist gut vernetzt und leicht zugänglich, dass
      es eine Schande wäre, dieses Material nicht wach und offenen Sinnes zu
      gebrauchen.“ (Matthias Käther, 2004)
    \end{itemize}
  \item {Leipzig Data} ist mehr als {Leipzig Open Data}, letzteres
    aber dessen Kern.
  \item {Leipzig Open Data} ist viel mehr als {Offene Daten der
    Stadt}, wir müssen (inhaltlich) nicht auf die Stadtverwaltung warten, um
    loszulegen.
  \end{itemize}
\end{frame}

\begin{frame}{Leipzig Data – Die Vision}{}
  \begin{itemize}
  \item Im Mittelpunkt von Leipzig Data stehen nicht die Daten (bzw.\ Worte)
    selbst, sondern die Menschen und deren Motive, die auf dieser Grundlage
    {planen} und {handeln}.
    \begin{itemize}
    \item[] \small\it Letztlich geht es dabei (noch immer) um die Verabredung
      Freier Bürger zu kooperativem Handeln. Leipzig Data reduziert diese Frage
      auf die {Unterstützung der Freien Rede Freier Bürger über die sie
        betreffenden Belange}.
    \end{itemize}
  \item Seit dem Turmbau zu Babel war nie so viel Möglichkeit, die Freie Rede
    Freier Bürger zur gemeinsamen Gestaltung unserer Mitwelt zu befördern, wie
    im heutigen Zeitalter einer nun auch technisch vernetzten Welt.
  \item Aber auch die Herausforderungen waren seit dem Turmbau zu Babel nie so
    groß wie heute.
  \end{itemize}
\end{frame}

\begin{frame}{Der Turmbau zu Babel}{}\small
  \begin{flushright}\small
    Der Turmbau zu Babel (Gen 11, 1–9)
  \end{flushright}
  \begin{itemize}
  \item Die Sprachverwirrung reicht bis heute fort – wir leben zusammen, ohne
    uns ausreichend zu verstehen oder uns gar in einem solchen Umfang zu
    gemeinsamem Handeln zu verabreden, wie es die Herausforderungen der Zeit
    eigentlich erfordern.
  \item Es geht also um Sprache, die Fähigkeit, unsere Ausdrucksmöglichkeiten
    zu erweitern und zu präzisieren {\ldots}
  \item Im Zentrum von Leipzig Data stehen die technische, inhaltliche und
    soziale Dimension der Weiterentwicklung unserer eigenen Sprache.
  \end{itemize}
  Mehr zum Zusammenhang von Technik und Sprache in meinem Beitrag
  {Stadtökologie und Semantic Web} zur Konferenz „Stadtökologie 2013“,
  \url{http://hg-graebe.de/EigeneTexte/LIFIS-16.pdf}.
\end{frame}

\begin{frame}{Leipziger Geschichten}{}
Leipziger Geschichten entstehen und werden fortgeschrieben in der – privaten
wie öffentlichen – Verantwortung
\begin{itemize}
\item sehr verschiedener Akteure
\item mit sehr unterschiedlichen Arbeitsschwerpunkten
\item und sehr verschiedenen Motivationen. 
\end{itemize}
Die regionalen {Potenziale digitaler Kooperation} nur können dann gehoben
werden können, wenn es gelingt, die Kooperations{fähigkeit} dieser Akteure
technisch, inhaltlich, sozial und politisch weiter voran zu bringen.
\medskip

Im Zentrum der Bemühungen muss dabei der {Prozess der Entwicklung einer
  gemeinsamen Sprache} stehen. Diesen Prozess gilt es auch technisch im
digitalen Umfeld zu verankern.
\end{frame}

\begin{frame}{Wie herangehen?}{}
API Leipzig – \url{http://www.apileipzig.de} – meint dazu
\begin{itemize}
\item API.LEIPZIG stellt öffentliche Daten flexibel zur Verfügung. Kreative
  Anwendungen machen diese Daten sichtbar und verständlich.
\item API.LEIPZIG fördert Vernetzung und Sichtbarkeit. Leipzigs kreatives
  Potenzial wird stärker gebündelt und vermarktet.
\item API.LEIPZIG schafft wirtschaftliche Perspektiven und Transparenz.
  Leipzig profitiert von einer positiven Entwicklung der Kultur- und
  Kreativwirtschaft.
\end{itemize}
Unsere Erfahrungen {agiler} Modellierung besagen – Sprache entsteht beim
Sprechen.
\end{frame}

\begin{frame}{Wie herangehen?}{}
Mit {Leipzig Data} nehmen wir den Ball von API Leipzig auf, haben aber
noch einmal über die Grundlagen nachgedacht.\medskip

Eine {offene} Diskussion über eine {offene} Stadtgesellschaft
erfordert die {offene} Verfügbarkeit grundsätzlicher Ideen und Fakten, die
diese Stadtgesellschaft konstituieren oder konstituieren sollen.
\begin{itemize}
\item[] Freie Bürger müssen insbesondere Freien Zugang zum {Bauplan ihres
  eigenen Hauses} haben.
\end{itemize}
Kooperatives Handeln steht im Spannungsfeld zwischen öffentlicher
Wirkung und privater Verantwortung. Für uns spielt eine zentrale Rolle,
dieses Spannungsfeld auch sprachlich auszuloten. Urteile und Verant-
wortlichkeiten für Urteile müssen klar sichtbar werden.\medskip

Eine offene Gesellschaft lebt zentral von privatem Engagement, das für
uns nur als verantwortungsbeladenes Engagement denkbar ist.
\end{frame}

\begin{frame}{Leipzig Data und Leipzig Open Data}{}
Leipzig Data als die kontroverse, spannungsgeladene, widersprüchliche
Gesamtheit der Worte Leipziger Geschichten.\medskip

Leipzig Open Data als allgemeiner, öffentlicher, konsensual befestigter Teil
davon.  Der schrittweise Aufbau eines solchen Datenbestandes – also eines
entsprechenden digital verfügbaren Wortschatzes – im Namensraum
\texttt{leipzig-data.de/Data} steht im Zentrum der Bemühungen von Leipzig Open
Data. \medskip

Dieser öffentliche Datenbestand ist kein Selbstzweck, sondern muss sich immer
daran bewähren, in welchem Umfang er privates, auch geschäftliches Engagement
zu unterstützen in der Lage ist.
\end{frame}

\begin{frame}{Wo stehen wir?}{}
  \begin{itemize}
  \item Vorarbeiten von API Leipzig (Orte und Adressen in Leipzig, Übersicht
    über Medienunternehmen, Event-Daten) sowie aus dem MINT- Netzwerk
    (Angebote, MINT-Orte, Träger, einige Bereiche der Jugendhilfe) sind
    weitgehend aufgearbeitet, in die Strukturen von Leipzig Data integriert,
    weiter angereichert und fortgeschrieben.
    \begin{itemize}
    \item Unterstützung durch das Kurzzeitprojekt Leipzig Open Data im Rahmen
      der städtischen Ausschreibung Open Innovation.
    \item Weitere Konsolidierung mit geringen Ressourcen nach Auslaufen der
      Förderung Ende April 2013.
    \end{itemize}
  \item Open Street Map, LinkedGeoData usw. offerieren weitere Informationen
    über Orte in Leipzig.
  \item Erschließung dieser Datenbestände mit den verfügbaren geringen
    Ressourcen steht im Zentrum aktueller Aktivitäten.
  \end{itemize}
\end{frame}

\begin{frame}{Wo stehen wir?}{}
Im Web verfügbar
\begin{itemize}
\item \url{http://www.leipzig-data.de} – Das Leipzig Data Portal.
\item \url{http://www.leipzig-data.de/Data} – RDF Data Store 
\item \url{http://leipzig-data.de:8890/sparql} – SPARQL Endpunkt für Anfragen
  auf den Daten.
\item \url{https://github.com/LeipzigData} – github Organisationsaccount von
  Leipzig Data mit den Projekten
  \begin{itemize}
  \item RDFData – RDF-Datenbasis
  \item Tools – eine Reihe von Werkzeugen (für Entwickler)
  \item Publications -- Veröffentlichungen und Präsentationen zum Projekt
  \end{itemize}
\item \url{http://www.leipzig-data.de/Backups} – wöchentliche Dumps der
  Daten.
\item \url{http://www.leipzig-data.de/ld-seminar} – Arbeitsseminar
  \emph{Anwendungen Semantischer Technologien}.
\end{itemize}
\end{frame}
\end{document}


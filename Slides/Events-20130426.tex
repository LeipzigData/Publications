\documentclass{beamer}
\usepackage{ngerman}
\usepackage[utf8]{inputenc}
% Designelemente
\usetheme{Boadilla}
\beamertemplatenavigationsymbolsempty
\setbeamertemplate{frametitle}
{%
\vspace{-0.165ex}

\begin{beamercolorbox}[wd=\paperwidth,dp=1ex, ht=4.5ex, sep=0.5ex,
    colsep*=0pt]{frametitle}
  \usebeamerfont{frametitle}   \strut \insertframetitle  \hfill 
  \raisebox{0ex}[0pt][-\ht\strutbox ]{
    \begin{minipage}[b]{.4\textwidth}\raggedleft \tiny Leipzig Open Data
      Initiative\\ Gefördert von der Stadt Leipzig
  \end{minipage}}
\end{beamercolorbox}%
}%

\title[Eventsframework]{Teilprojekt Eventsframework}

\subtitle{Bericht zum Abschlusstreffen des Projekts\\ „Leipzig Open Data
  Initiative“}

\author[Nareike]{Andreas Nareike}

\institute[Uni Leipzig]{Leipzig Open Data Team\\
  \texttt{http://leipzig-data.de}} 

\date{Leipzig, 26. April 2013}
\begin{document}
\begin{frame}
\maketitle
\end{frame}
\begin{frame}{Hintergrund}{}

Teilprojekt, das sich im Sinne von Open Innovation in der Arbeit nach der
Ideenbörse herauskristallisiert hat
\begin{itemize} 
\item \textbf{Zielstellung:} Infrastruktur aufbauen, in die Event-Daten in
  einheitlichem Format aus verschiedenen Quellen und von verschiedenen Akteuren
  eingespeist werden und der Allgemeinheit zum Gebrauch zur Verfügung stehen. 
\item \textbf{Partner:} API-Leipzig, city-cult, MINT-Netzwerk Leipzig, Netzwerk
  Energie \& Umwelt
\end{itemize}
\end{frame}

\begin{frame}{Erreichte Ergebnisse}{}
\begin{itemize} 
\item Prototypische Lösung auf der Basis des Exhibit-Frameworks
  (\url{http://www.leipzig-data.de/widget/})
\item Synergien aus dem Teilprojekt Gelbe Seiten (Daten von Orten, Trägern,
  Adressen und Geokoordinaten)
\item Etablierte Kanäle zum MINT-Netzwerk Leipzig und zum Netzwerk Energie \&
  Umwelt $\to$ wöchentliches Update
\item Bereitstellung einer Datenschnittstelle (SPARQL,
  \url{http://www.leipzig-data.de/widget/sparql.php}) sowie einer Schnittstelle
  mittels API-Leipzig $\to$ Nutzung durch city:cult (Smartphone App)
\end{itemize}
\end{frame}
\end{document}
